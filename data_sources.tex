
\documentclass{article}
\usepackage[margin=1in]{geometry}
\usepackage{hyperref}
\begin{document}
\title{Data Provenance for Cleveland STEM Value Proposition}
\author{Automated Notebook}
\date{\today}
\maketitle

\section{Bureau of Labor Statistics}
\subsection{Local Area Unemployment Statistics (LAUS)}
State-level unemployment rates provide a comparable labor-market stress signal. Series IDs follow the documented pattern \texttt{LASST\{STATEFIPS\}000000000003}. Data are retrieved via the Public Data API v2 at \url{https://api.bls.gov/publicAPI/v2/timeseries/data/} using the provided API key.

\subsection{Occupational Employment and Wage Statistics (OEWS)}
Annual median wages by occupation and metro are sourced from the bulk file \texttt{oe.data.1.AllData} hosted at \url{https://download.bls.gov/pub/time.series/oe/}. The notebook filters 2023 annual data (period A01) for STEM-relevant occupations including software developers (15-1252), electrical engineers (17-2071), biomedical engineers (17-2031), clinical laboratory technologists (29-2010), and management analysts (13-1111).

\section{U.S. Census Bureau}
\subsection{American Community Survey (ACS) 1-year}
The ACS subject tables are accessed through \url{https://api.census.gov/data/2023/acs/acs1/subject}. The pulls include median gross rent (variable \texttt{S2503\_C01\_001E}) and median earnings for full-time, year-round workers in science and engineering occupations (table \texttt{S2401}; variables \texttt{S2401\_C02\_001E}, \texttt{S2401\_C04\_012E}, \texttt{S2401\_C04\_014E}, \texttt{S2401\_C04\_016E}). Queries specify the target CBSA using the \texttt{for=metropolitan statistical area/micropolitan statistical area:\{CBSA\}} parameter and the provided Census API key.

\section{Output Artifacts}
The notebook writes intermediate CSVs (e.g., \texttt{laus\_unemployment\_rates.csv}, \texttt{oews\_stem\_msa\_wages.csv}, \texttt{acs\_rent\_stem\_earnings.csv}) and a consolidated \texttt{master\_real\_world\_metrics.csv} file used for downstream visualization.

\end{document}
